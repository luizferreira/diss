%% abtex2-modelo-trabalho-academico.tex, v-1.7.1 laurocesar
%% Copyright 2012-2013 by abnTeX2 group at http://abntex2.googlecode.com/ 
%%
%% This work may be distributed and/or modified under the
%% conditions of the LaTeX Project Public License, either version 1.3
%% of this license or (at your option) any later version.
%% The latest version of this license is in
%%   http://www.latex-project.org/lppl.txt
%% and version 1.3 or later is part of all distributions of LaTeX
%% version 2005/12/01 or later.
%%
%% This work has the LPPL maintenance status `maintained'.
%% 
%% The Current Maintainer of this work is the abnTeX2 team, led
%% by Lauro César Araujo. Further information are available on 
%% http://abntex2.googlecode.com/
%%
%% This work consists of the files abntex2-modelo-trabalho-academico.tex,
%% abntex2-modelo-include-comandos and abntex2-modelo-references.bib
%%

% ------------------------------------------------------------------------
% ------------------------------------------------------------------------
% abnTeX2: Modelo de Trabalho Academico (tese de doutorado, dissertacao de
% mestrado e trabalhos monograficos em geral) em conformidade com 
% ABNT NBR 14724:2011: Informacao e documentacao - Trabalhos academicos -
% Apresentacao
% ------------------------------------------------------------------------
% ------------------------------------------------------------------------

\documentclass[
    % -- opções da classe memoir --
    12pt,               % tamanho da fonte
    openright,          % capítulos começam em pág ímpar (insere página vazia caso preciso)
    twoside,            % para impressão em verso e anverso. Oposto a oneside
    a4paper,            % tamanho do papel. 
    % -- opções da classe abntex2 --
    %chapter=TITLE,     % títulos de capítulos convertidos em letras maiúsculas
    %section=TITLE,     % títulos de seções convertidos em letras maiúsculas
    %subsection=TITLE,  % títulos de subseções convertidos em letras maiúsculas
    %subsubsection=TITLE,% títulos de subsubseções convertidos em letras maiúsculas
    % -- opções do pacote babel --
    english,            % idioma adicional para hifenização
    french,             % idioma adicional para hifenização
    spanish,            % idioma adicional para hifenização
    brazil,             % o último idioma é o principal do documento
    ]{abntex2}


% ---
% PACOTES
% ---

% ---
% Pacotes fundamentais 
% ---
\usepackage{cmap}               % Mapear caracteres especiais no PDF
\usepackage{lmodern}            % Usa a fonte Latin Modern          
\usepackage[T1]{fontenc}        % Selecao de codigos de fonte.
\usepackage[utf8]{inputenc}     % Codificacao do documento (conversão automática dos acentos)
\usepackage{lastpage}           % Usado pela Ficha catalográfica
\usepackage{indentfirst}        % Indenta o primeiro parágrafo de cada seção.
\usepackage{color}              % Controle das cores
\usepackage{graphicx}           % Inclusão de gráficos
% ---
        
% ---
% Pacotes adicionais, usados apenas no âmbito do Modelo Canônico do abnteX2
% ---
\usepackage{lipsum}             % para geração de dummy text
% ---

% ---
% Pacotes de citações
% ---
\usepackage[brazilian,hyperpageref]{backref}     % Paginas com as citações na bibl
\usepackage[alf]{abntex2cite}   % Citações padrão ABNT

% --- 
% CONFIGURAÇÕES DE PACOTES
% --- 

% ---
% Configurações do pacote backref
% Usado sem a opção hyperpageref de backref
\renewcommand{\backrefpagesname}{Citado na(s) página(s):~}
% Texto padrão antes do número das páginas
\renewcommand{\backref}{}
% Define os textos da citação
\renewcommand*{\backrefalt}[4]{
    \ifcase #1 %
        Nenhuma citação no texto.%
    \or
        Citado na página #2.%
    \else
        Citado #1 vezes nas páginas #2.%
    \fi}%
% ---


% ---
% Informações de dados para CAPA e FOLHA DE ROSTO
% ---
\titulo{Projeto e Avaliação de um Sistema de Informação que Concilia Aumento
de Produtividade com Bem-estar em um Ambiente Organizacional Orientado a Projetos}
\autor{Luiz Gustavo da Fonseca Ferreiro}
\local{Belo Horizonte, Brasil}
\data{8 de Setembro de 2013}
\orientador{Prof. Dr. Marcello Peixoto Bax}
% \coorientador{Equipe \abnTeX}
\instituicao{%
  Universidade Federal de Minas Gerais -- UFMG
  % \par
  % Escola de Ci
  \par
  Programa de Pós-Graduação em Ciência da Informação}
\tipotrabalho{Dissertação (Mestrado)}
% O preambulo deve conter o tipo do trabalho, o objetivo, 
% o nome da instituição e a área de concentração 
\preambulo{Projeto apresentado como requisito de qualificação no Programa de Pós-Graduação
em Ciência da Informação.}
% ---


% ---
% Configurações de aparência do PDF final

% alterando o aspecto da cor azul
\definecolor{blue}{RGB}{41,5,195}

% informações do PDF
\makeatletter
\hypersetup{
        %pagebackref=true,
        pdftitle={\@title}, 
        pdfauthor={\@author},
        pdfsubject={\imprimirpreambulo},
        pdfcreator={LaTeX with abnTeX2},
        pdfkeywords={abnt}{latex}{abntex}{abntex2}{trabalho acadêmico}, 
        colorlinks=true,            % false: boxed links; true: colored links
        linkcolor=blue,             % color of internal links
        citecolor=blue,             % color of links to bibliography
        filecolor=magenta,              % color of file links
        urlcolor=blue,
        bookmarksdepth=4
}
\makeatother
% --- 

% --- 
% Espaçamentos entre linhas e parágrafos 
% --- 

% O tamanho do parágrafo é dado por:
\setlength{\parindent}{1.3cm}

% Controle do espaçamento entre um parágrafo e outro:
\setlength{\parskip}{0.2cm}  % tente também \onelineskip

% ---
% compila o indice
% ---
\makeindex
% ---

% ----
% Início do documento
% ----
\begin{document}

% Retira espaço extra obsoleto entre as frases.
\frenchspacing 

% ----------------------------------------------------------
% ELEMENTOS PRÉ-TEXTUAIS
% ----------------------------------------------------------
% \pretextual

% ---
% Capa
% ---
\imprimircapa
% ---

% ---
% Folha de rosto
% (o * indica que haverá a ficha bibliográfica)
% ---
\imprimirfolhaderosto*
% ---

% ---
% Inserir a ficha bibliografica
% ---

% Isto é um exemplo de Ficha Catalográfica, ou ``Dados internacionais de
% catalogação-na-publicação''. Você pode utilizar este modelo como referência. 
% Porém, provavelmente a biblioteca da sua universidade lhe fornecerá um PDF
% com a ficha catalográfica definitiva após a defesa do trabalho. Quando estiver
% com o documento, salve-o como PDF no diretório do seu projeto e substitua todo
% o conteúdo de implementação deste arquivo pelo comando abaixo:
%
% \begin{fichacatalografica}
%     \includepdf{fig_ficha_catalografica.pdf}
% \end{fichacatalografica}
\begin{fichacatalografica}
    \vspace*{\fill}                 % Posição vertical
    \hrule                          % Linha horizontal
    \begin{center}                  % Minipage Centralizado
    \begin{minipage}[c]{12.5cm}     % Largura
    
    \imprimirautor
    
    \hspace{0.5cm} \imprimirtitulo  / \imprimirautor. --
    \imprimirlocal, \imprimirdata-
    
    \hspace{0.5cm} \pageref{LastPage} p. : il. (algumas color.) ; 30 cm.\\
    
    \hspace{0.5cm} \imprimirorientadorRotulo~\imprimirorientador\\
    
    \hspace{0.5cm}
    \parbox[t]{\textwidth}{\imprimirtipotrabalho~--~\imprimirinstituicao,
    \imprimirdata.}\\
    
    \hspace{0.5cm}
        1. Felicidade.
        2. Bem-estar.
        3. Produtividade.
        4. Sistema de Informação.
        4. Gestão do Conhecimento.
        I. Marcello Peixoto Bax.
        II. Universidade Federal de Minas Gerais.
        III. PPGCI.
        IV. Projeto e Avaliação de um Sistema de Informação que Concilia Aumento
de Produtividade com Bem-estar em um Ambiente Organizacional Orientado a Projetos\\            
    
    \hspace{8.75cm} CDU 02:141:005.7\\
    
    \end{minipage}
    \end{center}
    \hrule
    \pagebreak % está aqui porque se tirar os agradecimentos a epigrafe sobe.
\end{fichacatalografica}
% ---

% % ---
% % Inserir errata
% % ---
% \begin{errata}
% Elemento opcional da \citeonline[4.2.1.2]{NBR14724:2011}. Exemplo:

% \vspace{\onelineskip}

% FERRIGNO, C. R. A. \textbf{Tratamento de neoplasias ósseas apendiculares com
% reimplantação de enxerto ósseo autólogo autoclavado associado ao plasma
% rico em plaquetas}: estudo crítico na cirurgia de preservação de membro em
% cães. 2011. 128 f. Tese (Livre-Docência) - Faculdade de Medicina Veterinária e
% Zootecnia, Universidade de São Paulo, São Paulo, 2011.

% \begin{table}[htb]
% \center
% \footnotesize
% \begin{tabular}{|p{1.4cm}|p{1cm}|p{3cm}|p{3cm}|}
%   \hline
%    \textbf{Folha} & \textbf{Linha}  & \textbf{Onde se lê}  & \textbf{Leia-se}  \\
%     \hline
%     1 & 10 & auto-conclavo & autoconclavo\\
%    \hline
% \end{tabular}
% \end{table}

% \end{errata}
% ---

% % ---
% % Inserir folha de aprovação
% % ---

% % Isto é um exemplo de Folha de aprovação, elemento obrigatório da NBR
% % 14724/2011 (seção 4.2.1.3). Você pode utilizar este modelo até a aprovação
% % do trabalho. Após isso, substitua todo o conteúdo deste arquivo por uma
% % imagem da página assinada pela banca com o comando abaixo:
% %
% % \includepdf{folhadeaprovacao_final.pdf}
% %
% \begin{folhadeaprovacao}

%   \begin{center}
%     {\ABNTEXchapterfont\large\imprimirautor}

%     \vspace*{\fill}\vspace*{\fill}
%     {\ABNTEXchapterfont\bfseries\Large\imprimirtitulo}
%     \vspace*{\fill}
    
%     \hspace{.45\textwidth}
%     \begin{minipage}{.5\textwidth}
%         \imprimirpreambulo
%     \end{minipage}%
%     \vspace*{\fill}
%    \end{center}
    
%    Trabalho aprovado. \imprimirlocal, 24 de novembro de 2012:

%    \assinatura{\textbf{\imprimirorientador} \\ Orientador} 
%    \assinatura{\textbf{Professor} \\ Convidado 1}
%    \assinatura{\textbf{Professor} \\ Convidado 2}
%    %\assinatura{\textbf{Professor} \\ Convidado 3}
%    %\assinatura{\textbf{Professor} \\ Convidado 4}
      
%    \begin{center}
%     \vspace*{0.5cm}
%     {\large\imprimirlocal}
%     \par
%     {\large\imprimirdata}
%     \vspace*{1cm}
%   \end{center}
  
% \end{folhadeaprovacao}
% % ---

% ---
% Dedicatória
% ---
% \begin{dedicatoria}
%    \vspace*{\fill}
%    \centering
%    \noindent
%    \textit{Este trabalho é dedicado às crianças adultas que, sonharam \\
%    quando pequenas, e não desistiram dos seus sonhos de criança.} \vspace*{\fill}
% \end{dedicatoria}
% ---

% ---
% Agradecimentos
% ---
% \begin{agradecimentos}
% Os agradecimentos principais são direcionados à Gerald Weber, Miguel Frasson,
% Leslie H. Watter, Bruno Parente Lima, Flávio de Vasconcellos Corrêa, Otavio Real
% Salvador, Renato Machnievscz\footnote{Os nomes dos integrantes do primeiro
% projeto abn\TeX\ foram extraídos de
% \url{http://codigolivre.org.br/projects/abntex/}} e todos aqueles que
% contribuíram para que a produção de trabalhos acadêmicos conforme
% as normas ABNT com \LaTeX\ fosse possível.

% Agradecimentos especiais são direcionados ao Centro de Pesquisa em Arquitetura
% da Informação\footnote{\url{http://www.cpai.unb.br/}} da Universidade de
% Brasília (CPAI), ao grupo de usuários
% \emph{latex-br}\footnote{\url{http://groups.google.com/group/latex-br}} e aos
% novos voluntários do grupo
% \emph{\abnTeX}\footnote{\url{http://groups.google.com/group/abntex2} e
% \url{http://abntex2.googlecode.com/}}~que contribuíram e que ainda
% contribuirão para a evolução do \abnTeX.

% \end{agradecimentos}
% ---

% ---
% Epígrafe
% ---
% \begin{epigrafe}
%     \vspace*{\fill}
%     \begin{flushright}
%         \textit{``Não vos amoldeis às estruturas deste mundo, \\
%         mas transformai-vos pela renovação da mente, \\
%         a fim de distinguir qual é a vontade de Deus: \\
%         o que é bom, o que Lhe é agradável, o que é perfeito.\\
%         (Bíblia Sagrada, Romanos 12, 2)}
%     \end{flushright}
% \end{epigrafe}
% ---

% ---
% RESUMOS
% ---

% resumo em português
\begin{resumo}
 Segundo a \citeonline[3.1-3.2]{NBR6028:2003}, o resumo deve ressaltar o
 objetivo, o método, os resultados e as conclusões do documento. A ordem e a extensão
 destes itens dependem do tipo de resumo (informativo ou indicativo) e do
 tratamento que cada item recebe no documento original. O resumo deve ser
 precedido da referência do documento, com exceção do resumo inserido no
 próprio documento. (\ldots) As palavras-chave devem figurar logo abaixo do
 resumo, antecedidas da expressão Palavras-chave:, separadas entre si por
 ponto e finalizadas também por ponto.

 \vspace{\onelineskip}
    
 \noindent
 \textbf{Palavras-chaves}: latex. abntex. editoração de texto.
\end{resumo}

% resumo em inglês
\begin{resumo}[Abstract]
 \begin{otherlanguage*}{english}
   This is the english abstract.

   \vspace{\onelineskip}
 
   \noindent 
   \textbf{Key-words}: latex. abntex. text editoration.
 \end{otherlanguage*}
\end{resumo}

% resumo em francês 
\begin{resumo}[Résumé]
 \begin{otherlanguage*}{french}
    Il s'agit d'un résumé en français.
 
   \vspace{\onelineskip}
 
   \noindent
   \textbf{Mots-clés}: latex. abntex. publication de textes.
 \end{otherlanguage*}
\end{resumo}

% resumo em espanhol
\begin{resumo}[Resumen]
 \begin{otherlanguage*}{spanish}
   Este es el resumen en español.
  
   \vspace{\onelineskip}
 
   \noindent
   \textbf{Palabras clave}: latex. abntex. publicación de textos.
 \end{otherlanguage*}
\end{resumo}
% ---

% ---
% inserir lista de ilustrações
% ---
\pdfbookmark[0]{\listfigurename}{lof}
\listoffigures*
\cleardoublepage
% ---

% ---
% inserir lista de tabelas
% ---
\pdfbookmark[0]{\listtablename}{lot}
\listoftables*
\cleardoublepage
% ---

% ---
% inserir lista de abreviaturas e siglas
% ---
\begin{siglas}
  \item[Fig.] Area of the $i^{th}$ component
  \item[456] Isto é um número
  \item[123] Isto é outro número
  \item[lauro cesar] este é o meu nome
\end{siglas}
% ---

% ---
% inserir lista de símbolos
% ---
\begin{simbolos}
  \item[$ \Gamma $] Letra grega Gama
  \item[$ \Lambda $] Lambda
  \item[$ \zeta $] Letra grega minúscula zeta
  \item[$ \in $] Pertence
\end{simbolos}
% ---

% ---
% inserir o sumario
% ---
\pdfbookmark[0]{\contentsname}{toc}
\tableofcontents*
\cleardoublepage
% ---



% ----------------------------------------------------------
% ELEMENTOS TEXTUAIS
% ----------------------------------------------------------
\textual

% ----------------------------------------------------------
% Introdução
% ----------------------------------------------------------

\chapter*[Introdução]{Introdução}
\addcontentsline{toc}{chapter}{Introdução}

Segundo pesquisas, cerca de 40\% dos trabalhadores não se consideram felizes em seu trabalho. Esse é um dado preocupante, uma vez que um indivíduo normal passa cerca de 1/3 da sua vida trabalhando. Ainda nesse contexto, estudos têm mostrado uma relação positiva entre o bem estar de um indivíduo e a sua produtividade, ou seja, não apenas o próprio indivíduo se beneficia de um ambiente de trabalho mais saudável, mas também a organização para qual esse indivíduo trabalha [REF?]. Devido a sua importância, a relação entre felicidade e desempenho é uma questão que despertou a fascinação de desde filósofos sociais (como Russeau [em que obra?]) até executivos e pesquisadores na área de comportamento organizacional.

Nos últimos 60 anos, a psicologia tem trabalho sobre um modelo de doença, que consiste em encontrar o que há de errado com as pessoas. Mais recentemente, notadamente na última década, temos visto ganhar força um movimento chamado de psicologia positiva, que estuda o que de fato faz a vida valer a pena. Essa nova ciência é baseada na crença de que a psicologia deveria estar tão preocupada em desenvolver os pontos fortes quanto consertar os danos. A psicologia positiva trás novas percepções à questão da relação entre produtividade e felicidade no trabalho, notadamente através de conceitos como engajamento, significado na vida e fluxo. (Seligman, 2012)

Vários estudos têm abordado essa relação entre felicidade e produtividade. Psicólogos têm mostrado que a sensação de emoções positivas influenciam positivamente a capacidade de decisão e geração de inovação (Isen, 2000), melhora a recuperação da memória (Isen et al. 1978; Teasdale e Fogarty 1979, e leva a níveis maiores de altruísmo (Isen e Simmonds 1978). Isen e Reeve (2005) mostram que a afeição positiva levam as pessoas a mudar a alocação de seu tempo para tarefas mais interessantes, e apesar disso, conseguem manter níveis similares aos outros nas tarefas menos interessantes. Isso é um indício de que indivíduos mais felizes têm a capacidade de realizar tarefas repetitivas mais eficientemente.

A utilização de sistemas de informação nas últimas décadas tem abrangido as mais diversas áreas. Desde a prospecção de petróleo em águas profundas ao auxílio nas tomadas de decisão por profissionais de saúde, esses sistemas têm apoiado as pessoas nas mais diversas atividades. Porém, apesar da importância dada a questão da relação entre felicidade e produtividade no trabalho, não há estudos significativos que pesquisam como um sistema de informação pode auxiliar nesse equilíbrio tão procurado empresários, pesquisadores e filósofos. ISSO PODE SE EXPLICAR PELO FATO DAS PESSOAS NÃO IMAGINAREM QUE UM SISTEMA DE INFORMAÇÃO POSSA RESULTAR EM ALGUMA IMPLICAÇÃO EFETIVA NO TRATAMENTO DA QUESTÃO. 

A felicidade possui valor por si só e é provavelmente o objetivo mais perseguido pelas pessoas. Conciliar isso a um aumento de performance no trabalho é considerado o santo graal da pesquisa em comportamento organizacional [REF?]. 

O projeto de um sistema de informação designado a esse fim é um tópico ainda a ser explorado. Nesse contexto, surge a seguinte questão de pesquisa: é possível desenvolver um sistema de informação que auxilie no aumento da produtividade e ao mesmo tempo a satisfação das pessoas em um ambiente organizacional orientado a projetos?


% ----------------------------------------------------------
% PARTE - referencial teórico
% ----------------------------------------------------------

% ----------------------------------------------------------
% Parte de referencial teórico
% ----------------------------------------------------------
\part{Referencial Teórico}

% ---
% Capitulo de revisão de literatura
% ---
\chapter{Lorem ipsum dolor sit amet}

% ---
\section{Aliquam vestibulum fringilla lorem}
% ---

\lipsum[1]

\lipsum[2-3]

% ----------------------------------------------------------
% PARTE - metodologia
% ----------------------------------------------------------

% ----------------------------------------------------------
% Resultados
% ----------------------------------------------------------
\part{Metodologia}

Nesta pesquisa será realizado um estudo de caso. Basicamente, o objetivo desse
estudo de caso é testar a hipótese de que o sistema de informação projetado
possui um efeito benéfico na concentração das pessoas que o utilizam. Para 
possibilitar esse estudo, serão utilizadas algumas ferramentas e métodos, que serão descritos
nas próximas seções. As ferramentas que serão utilizadas:

\begin{itemize}
    \item O sistema de informação;
    \item Um programa para auxiliar a medição da concentração das pessoas;
\end{itemize}



% ---
% primeiro capitulo da Metodologia.
% ---
\chapter{Lectus lobortis condimentum}

% ---
\section{Vestibulum ante ipsum primis in faucibus orci luctus et ultrices
posuere cubilia Curae}
% ---

\lipsum[21-22]



% ---
% Finaliza a parte no bookmark do PDF, para que se inicie o bookmark na raiz
% ---
\bookmarksetup{startatroot}% 
% ---

% % ---
% % Conclusão
% % ---
% \chapter*[Conclusão]{Conclusão}
% \addcontentsline{toc}{chapter}{Conclusão}

% \lipsum[31-33]

% ----------------------------------------------------------
% ELEMENTOS PÓS-TEXTUAIS
% ----------------------------------------------------------
\postextual


% ----------------------------------------------------------
% Referências bibliográficas
% ----------------------------------------------------------
\bibliography{chapters/0bibliografia}

% ----------------------------------------------------------
% Glossário
% ----------------------------------------------------------
%
% Consulte o manual da classe abntex2 para orientações sobre o glossário.
%
%\glossary

% ----------------------------------------------------------
% Apêndices
% ----------------------------------------------------------

% ---
% Inicia os apêndices
% ---
% \begin{apendicesenv}

% % Imprime uma página indicando o início dos apêndices
% \partapendices

% % ----------------------------------------------------------
% \chapter{Quisque libero justo}
% % ----------------------------------------------------------

% \lipsum[50]

% % ----------------------------------------------------------
% \chapter{Nullam elementum urna vel imperdiet sodales elit ipsum pharetra ligula
% ac pretium ante justo a nulla curabitur tristique arcu eu metus}
% % ----------------------------------------------------------
% \lipsum[55-57]

% \end{apendicesenv}
% ---


% ----------------------------------------------------------
% Anexos
% ----------------------------------------------------------

% ---
% Inicia os anexos
% ---
% \begin{anexosenv}

% % Imprime uma página indicando o início dos anexos
% \partanexos

% % ---
% \chapter{Morbi ultrices rutrum lorem.}
% % ---
% \lipsum[30]

% % ---
% \chapter{Cras non urna sed feugiat cum sociis natoque penatibus et magnis dis
% parturient montes nascetur ridiculus mus}
% % ---

% \lipsum[31]

% % ---
% \chapter{Fusce facilisis lacinia dui}
% % ---

% \lipsum[32]

% \end{anexosenv}

%---------------------------------------------------------------------
% INDICE REMISSIVO
%---------------------------------------------------------------------

\printindex

\end{document}
