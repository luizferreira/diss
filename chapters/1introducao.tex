\chapter*[Introdução]{Introdução}
\addcontentsline{toc}{chapter}{Introdução}

Segundo pesquisas, cerca de 40\% dos trabalhadores não se consideram felizes em seu trabalho. Esse é um dado preocupante, uma vez que um indivíduo normal passa cerca de 1/3 da sua vida trabalhando. Ainda nesse contexto, estudos têm mostrado uma relação positiva entre o bem estar de um indivíduo e a sua produtividade, ou seja, não apenas o próprio indivíduo se beneficia de um ambiente de trabalho mais saudável, mas também a organização para qual esse indivíduo trabalha [REF?]. Devido a sua importância, a relação entre felicidade e desempenho é uma questão que despertou a fascinação de desde filósofos sociais (como Russeau [em que obra?]) até executivos e pesquisadores na área de comportamento organizacional.

Nos últimos 60 anos, a psicologia tem trabalho sobre um modelo de doença, que consiste em encontrar o que há de errado com as pessoas. Mais recentemente, notadamente na última década, temos visto ganhar força um movimento chamado de psicologia positiva, que estuda o que de fato faz a vida valer a pena. Essa nova ciência é baseada na crença de que a psicologia deveria estar tão preocupada em desenvolver os pontos fortes quanto consertar os danos. A psicologia positiva trás novas percepções à questão da relação entre produtividade e felicidade no trabalho, notadamente através de conceitos como engajamento, significado na vida e fluxo. (Seligman, 2012)

Vários estudos têm abordado essa relação entre felicidade e produtividade. Psicólogos têm mostrado que a sensação de emoções positivas influenciam positivamente a capacidade de decisão e geração de inovação (Isen, 2000), melhora a recuperação da memória (Isen et al. 1978; Teasdale e Fogarty 1979, e leva a níveis maiores de altruísmo (Isen e Simmonds 1978). Isen e Reeve (2005) mostram que a afeição positiva levam as pessoas a mudar a alocação de seu tempo para tarefas mais interessantes, e apesar disso, conseguem manter níveis similares aos outros nas tarefas menos interessantes. Isso é um indício de que indivíduos mais felizes têm a capacidade de realizar tarefas repetitivas mais eficientemente.

A utilização de sistemas de informação nas últimas décadas tem abrangido as mais diversas áreas. Desde a prospecção de petróleo em águas profundas ao auxílio nas tomadas de decisão por profissionais de saúde, esses sistemas têm apoiado as pessoas nas mais diversas atividades. Porém, apesar da importância dada a questão da relação entre felicidade e produtividade no trabalho, não há estudos significativos que pesquisam como um sistema de informação pode auxiliar nesse equilíbrio tão procurado empresários, pesquisadores e filósofos. ISSO PODE SE EXPLICAR PELO FATO DAS PESSOAS NÃO IMAGINAREM QUE UM SISTEMA DE INFORMAÇÃO POSSA RESULTAR EM ALGUMA IMPLICAÇÃO EFETIVA NO TRATAMENTO DA QUESTÃO. 

A felicidade possui valor por si só e é provavelmente o objetivo mais perseguido pelas pessoas. Conciliar isso a um aumento de performance no trabalho é considerado o santo graal da pesquisa em comportamento organizacional [REF?]. 

O projeto de um sistema de informação designado a esse fim é um tópico ainda a ser explorado. Nesse contexto, surge a seguinte questão de pesquisa: é possível desenvolver um sistema de informação que auxilie no aumento da produtividade e ao mesmo tempo a satisfação das pessoas em um ambiente organizacional orientado a projetos?
