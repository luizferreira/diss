% ----------------------------------------------------------
% Parte de referencial teórico
% ----------------------------------------------------------
\part{Referencial Teórico}

% ---
% Capitulo de revisão de literatura
% ---
\chapter{Psicologia Positiva}

A psicologia positiva é um movimento recente dentro da ciência psicológica que visa fazer com que os psicólogos contemporâneos adotem \"uma visão mais aberta e apreciativa dos potenciais, das motivações e das capacidades humanas\". (Sheldon e King, 2001)

Embora as pessoas venham discutindo a questão da felicidade humana pelo menos desde a Grécia Antiga, a psicologia tem sido criticada por seu direcionamento preponderantemente voltado às questões de doença mental, em vez da \"sanidade\" mental. Vários psicólogos humanistas tais como Abraham Maslow, Carl Rogers e Erich Fromm, Carl Jung, desenvolveram teorias e práticas bem-sucedidas que envolvem a felicidade humana, a despeito da falta de evidência empírica sólida ao tempo em que desenvolveram seus trabalhos. Seus sucessores não deram seqüência à obra, enfatizando a fenomenologia e histórias de casos individuais. (Seligman, 2002)

Recentemente, as teorias de desenvolvimento humano desenvolvidas pelos psicólogos humanistas encontraram suporte empírico em estudos feitos por psicólogos positivos e humanistas, especialmente na área da teoria auto-determinante. (Patterson e Joseph, 2007). Pesquisadores empíricos nesse campo de estudo incluem Donald Clifton, Albert Bandura, Martin Seligman, Armindo Freitas-Magalhães, Ed Diener, Mihaly Csikszentmihalyi, C. R. Snyder, Christopher Peterson, Shelley Taylor, Barbara Fredrickson, Charles S. Carver, Michael F. Scheier e Jonathan Haidt.

Peterson e Seligman (2004) desenvolveram um sistema de classificação para os aspectos positivos, enfatizando as forças e o caráter denominado Values in Action (VIA) – Classification of Strengths and Virtues Manual (Tradução livre: Valores em ação - Manual de classificação de forças e virtudes). Nesse manual as forças foram divididas em características emocionais, cognitivas, relacionais e cívicas e em seis grupos de virtudes: sabedoria, coragem, humanidade, justiça, temperamento e transcendência.

Alguns dos principais fatores correlacionados com felicidade estudados foram (Paludo e Koller, 2007):

\begin{itemize}
    \item Amigos íntimos e presentes (Myers 2000)
    \item Fazer atividades voluntárias para desenvolvimento de si e de outros (Larson 2000)
    \item Estabelecer uma relação familiar de apoio e estímulo ao desenvolvimento de habilidades (Winner 2000)
    \item Relações saudáveis no ambiente de trabalho (Turner, Barling \& Zacharatos, 2002)
    \item Desenvolvimento de atividades com alto envolvimento (ou fluxo).
\end{itemize}

A psicologia da saúde e outras ciências da saúde também tem demostrado já a algumas décadas o interesse de mudar o foco do processo saúde-doença para o bem estar e medidas preventivas ao invés de se focar na doença quando ela já está presente e remediar como a medicina tradicional.

Na terapia cognitiva, o objetivo é ajudar as pessoas a mudar estilos negativos de pensamento como uma maneira de mudar como elas sentem. Essa abordagem tem sido bem sucedida, e mudar a maneira como pensamos sobre as pessoas, nosso futuro, e nós mesmos é parcialmente responsável por esse sucesso. A maneira como o processo de pensamento impacta em como nos sentimos varia consideravelmente de pessoa pra pessoa. A habilidade de desviar a nossa atenção da chamada conversa interna crônica ou nossos pensamentos é bastante benéfica ao nosso bem estar. Paralelamente, os objetivos da psicologia positiva, segundo o Dr. Martin Seligman - diretor do Centro de Psicologia Positiva da Universidade da Pensilvânia, incluem  possibilidades como: auxiliar famílias e escolas na educação de crianças e ambientes de trabalho que objetivam tanto o bem-estar quanto a alta produtividade. (Seligman, 2007)

\chapter{Bem-estar}

Segundo Aristóteles, toda a ação humana visa encontrar a felicidade (CITE). Já Nietzche acreditava que a ação humana era motivada pela busca do poder (CITE). Por fim, Freud pensava que a ação humana pretendia simplesmente evitar a ansiedade (CITE). Segundo Seligman (Florescer, CITE), todos esses grandes pensadores do passado cometeram o equívoco do monismo, pelo qual todas as as motivações humanas se resumem a apenas uma. Os monismos são \underline{acreditáveis} em ambientes com poucas variáveis, e por isso passam com sucesso no teste do bom senso, \underline{seguindo a lógica} da máxima filosófica de que a resposta mais simples é a melhor. Porém, quando as variáveis são insuficientes para explicar as ricas nuances do fenômeno em questão, nada é explicado.

Panoramico filosofico sobre a felicidade:
As 4 chaves (ou correntes) filosóficas para entender a felicidade:
Felicidade como busca pelo prazer; (emoções positivas de Seligman)
Felicidade como realização de desejos; (realização de Seligman)
Felicidade pela busca da razão (significado de Seligman)
Felicidade analisada sob o ponto de vista do sofrimento.

Kirkgard diz que existem três formas de enfrentar a angústia para ser feliz:
Estágio estético (achar que você será feliz sentindo coisas: comprando, transando, comendo, bebendo, etc) -> Insustentável: uma hora  não faz mais efeito e você vai voltar para a angústia.
Estágio ético: Achando que você é bom. \"Eu sou uma pessoa ética\". Você faz o que as pessoas dizem que é certo, você acredita que é \underline{o bonzinho}. (Segundo Pondé, não funciona)
Estágio religioso:
Achar que vai fugir da angústia seguindo a doutrin`a religiosa pura e simples.
Desiste de toda tentativa de ser feliz tendo sensações, ser feliz sendo legal e ser feliz seguindo uma doutrina e aposta em Deus.

Felicidade = Emoções positivas (CITE) http://www.youtube.com/watch?v=vZKwz6rigfU

O sentido popular mais utilizado para felicidade está fortemente amarrado a um estado de boa disposição. A emoção positiva é o sentido mais básico da felicidade. Isso é evidenciado por pesquisas (CITE) que mostram que o estado de ânimo em uma pessoa está determina mais de 70% da quantidade de satisfação com a vida que ela relata, e o julgamento que ela faz de como está sua vida nesse momento determina menos de 30%. 

Utilizamos o termo felicidade com diferentes significados em nosso cotidiano. Muitas vezes o usamos para caracterizar um estado em que se sente emoções como alegria e animação. Historicamente, a \underline{felicidade} não está intimamente atrelada a tais hedonismos, alegria e bom humor estão muito longe daquilo que Thomas Jefferson, por exemplo, declarou aos cidadães americanos como um direito a se perseguir. Essa \underline{definição} está mais longe ainda do que é concebido pela psicologia positiva. A psicologia positiva tem a ver com aquilo que os seres humanos escolhem por si mesmo. Muitas pessoas praticam meditação porque isso as fazem se sentir bem, escolhem realizar essa atividade por ela própria, não por qualquer outra motivação. Frequentemente escolhemos fazer o que nos fazem bem, mas nem sempre é assim, muitas vezes nossas decisões não têm a ver com o modo como nos sentimos. Certa vez deixei de ir a uma festa para assistir ao teatro da turma do meu sobrinho de 5 anos, muitas outra vezes deixei de sair com os amigos para escrever minha dissertação de mestrado. Esses tipos de decisão não são baseadas em como nos sentimos, meu sobrinho não é lá um grande ator e deixar de sair para me dedicar à minha dissertação certamente não me deixa alegre, minhas decisões foram baseadas principalmente no que faz sentido à minha vida.

\chapter{Fluxo}

Fluxo é um estado mental de operação onde o indivíduo realizando uma atividade está completamente imerso em um sentimento de foco energizado, envolvimento completo e diversão durante o processo de execução da atividade. Em sua essência, o fluxo é caracterizado pela completa absorção no que se está sendo feito. Proposto por Mihály Csikszentmihályi, esse conceito da psicologia positiva está sendo referenciado constantemente em vários campos do conhecimento. (Csikszentmihályi, 1997)

De acordo com Csikszentmihályi, o fluxo está bastante relacionado com motivação. É uma imersão na atividade que representa talvez a experiência suprema de colocar suas emoções a serviço do desempenho em alguma atividade. No fluxo, as emoções não são apenas contidas e canalizadas, mas positivas, energizadas e alinhadas com a tarefa em mãos. A marca do fluxo é um sentimento leve e espontâneo de alegria durante o realizar de uma atividade (Goleman, 1995), apesar do fluxo também ser descrito apenas como o ato de focar em uma atividade.

Os componentes de uma experiência de fluxo podem ser especificamente enumerados. Apesar de todos os componentes abaixo caracterizarem o estado de fluxo, não é necessária a presença de todas estas sensações para experienciar esse estado:

Objetivos claros (expectativas e regras são discerníveis).
Concentração e foco (um alto grau de concentração em um limitado campo de atenção).
Perda do sentimento de auto-consciência.
Sensação de tempo distorcida.
\emph{Feedback} direto e imediato (acertos e falhas no decurso da atividade são aparentes, podendo ser corrigidos se preciso).
Equilíbrio entre o nível de habilidade e de desafio (a atividade nunca é demasiadamente simples ou complicada).
A sensação de controle pessoal sobre a situação ou a atividade.
A atividade é em si recompensadora, não exigindo esforço algum.
Quando se encontram em estado de fluxo, as pessoas praticamente \"se tornam parte da atividade\" que estão praticando e a consciência é focada totalmente na atividade em si.

\chapter{Relação entre felicidade e produtividade}

Vários estudos têm pesquisado a relação entre felicidade e produtivividade, e muitos deles têm mostrado que a felicidade influencia positivamente no desempenho de um indivíduo. Oswald, Proto e Sgroi (2009) provê evidências que a felicidade aumenta a produtividade. Nesse estudo, foram realizados dois experimentos. No primeiro, uma amostra aleatória foi designada. Alguns indivíduos tiveram seus níveis de felicidade aumentado, enquanto os outros no grupo de controle não. Os indivíduos do primeiro grupo tiveram desempenho 12\% melhor nas tarefas designadas. Para testar a robustez e a natureza duradoura desse tipo de efeito, um experimento complementar foi realizado. Nesse último, foram estudados indivíduos que tiveram alguma espécie de problema que afetasse a sua felicidade (luto ou doença na família). Os resultados do segundo experimento corroboraram aqueles do primeiro experimento.


Um ambiente de trabalho que propicia boas relações interpessoais influencia significativamente na produtividade das pessoas. Um estudo realizado por Campion, Papper e Medsker (1996) com mais de 350 empregados em 60 unidades de negócio de uma companhia de serviços financeiros descobriu que o melhor previsor de sucesso de uma equipe era como os membros se sentiam em relação aos outros da equipe. Isso é especialmente importante para administradores, porque enquanto eles frequentemente tem pouco controle sobre a experiência e as habilidades das pessoas que são colocadas em seus times, eles têm controle efetivo sobre o nível de interação e harmonia entre os membros. Estudos têm mostrado que quanto mais os membros da equipe são encorajados a socializar e interagir, mais engajados eles se sentem, mais energia eles demonstram, e mais tempo eles conseguem manter-se focados em uma tarefa (Heaphy e Dutton, 2008). Resumindo, quanto mais os membros do time investem em coesão social, melhores serão os seus resultados no trabalho.


% % ---
% \section{Aliquam vestibulum fringilla lorem}
% % ---

% \lipsum[1]

% \lipsum[2-3]